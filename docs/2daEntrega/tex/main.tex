%%%%%%%%%%%%%%%%%%%%%%%%%%%%%%%%%%%%%%%%%%%%%%%%%%%%%%%%%%%%%%%%%%%%%%%%%%%%%%%
% Definici�n del tipo de documento.                                           %
% Posibles tipos de papel: a4paper, letterpaper, legalpapper                  %
% Posibles tama�os de letra: 10pt, 11pt, 12pt                                 %
% Posibles clases de documentos: article, report, book, slides                %
%%%%%%%%%%%%%%%%%%%%%%%%%%%%%%%%%%%%%%%%%%%%%%%%%%%%%%%%%%%%%%%%%%%%%%%%%%%%%%%
\documentclass[10pt, spanish, a4paper]{article}

%%%%%%%%%%%%%%%%%%%%%%%%%%%%%%%%%%%%%%%%%%%%%%%%%%%%%%%%%%%%%%%%%%%%%%%%%%%%%%%
% Los paquetes permiten ampliar las capacidades de LaTeX.                     %
%%%%%%%%%%%%%%%%%%%%%%%%%%%%%%%%%%%%%%%%%%%%%%%%%%%%%%%%%%%%%%%%%%%%%%%%%%%%%%%
\usepackage[spanish]{babel}     % Paquete para definir el idioma usado.
\usepackage[latin1]{inputenc}   % Define la codificaci�n de caracteres 
                                % (latin1 es ISO 8859-1)
%\usepackage[T1]{fontenc}        % Agrega caracteres extendidos al font
\usepackage{t1enc}
\usepackage{palatino}           % Cambia el font por omision a Palatino
\usepackage{graphicx}           % Paquete para inclusi�n de gr�ficos.
%%%%%%%%%%%%%%%%%%%%%%%%%%%%%%%%%%%%%%%%%%%%%%%%%%%%%%%%%%%%%%%%%%%%%%%%%%%%%%%

% T��tulo principal del documento.
\title{\textbf{The Speaker (2da Entrega)}}

% Informaci�n sobre los autores.
\author{    
            Juan Manuel Barrenche, \textit{Padr�n Nro. 86.152}                 \\
            \texttt{ snipperme@gmail.com }                                     \\
            Mart��n Fern�ndez, \textit{Padr�n Nro. 88.171}                      \\
            \texttt{ tinchof@gmail.com }                                       \\
            Marcos J. Medrano, \textit{Padr�n Nro. 86.729}                     \\
            \texttt{ marcosmedrano0@gmail.com }                                \\
            Federico Valido, \textit{Padr�n Nro. 82.490}                       \\
            \texttt{ fvalido@gmail.com }                                       \\ 
                                                                               \\
            \normalsize{Grupo Nro. 11 (YES)}                                   \\
            \normalsize{Ayudante: Renzo Navas}                         	       \\
            \normalsize{1er. Cuatrimestre de 2009}                             \\
            \normalsize{75.06 Organizaci�n de Datos - Titular: Arturo Servetto}\\
            \normalsize{Facultad de Ingenier�a, Universidad de Buenos Aires}   \\
       }
\date{Domingo 10 de Mayo de 2009}


% Comienzo del documento
\begin{document}

\maketitle                % Inserta el t�tulo.
\thispagestyle{empty}     % Quita el n�mero en la primer p�gina.

% Resumen que aparece en la primera p�gina (antes de la tabla de contenidos)
\begin{abstract}
El presente trabajo representa una aplicaci�n de los conceptos vistos durante
el m�dulo de \textit{Organizaci�n de Archivos} del curso de \textit{75.06 Organizaci�n de Datos} 
de la c�tedra Servetto.\\
\\
Consiste en implementar un sistema que permita la lectura de textos y la 
reproducci�n de los mismos en audio. El usuario carga los audios correspondientes
a cada termino encontrado en un archivo de texto. El sistema debe ser capaz de 
persistirlos, recuperarlos y de reproducir de un texto, cada termino individual 
en base a las grabaciones cargadas previamente. \\
Este documento ha sido desarrollado en \LaTeX.
\end{abstract}

\newpage
\tableofcontents        % Comentar si no se desea la Table Of Content.
\newpage

% Inclusi�n de archivos externos
\section{Librer�a de Documentos (DocumentLibrary)}
Este modulo representa la abstraccion de Documentos que utilizamos, dentro de el existen las clases para la persistencia de los mismos, la interfase de un documento y tres distintas implementaciones: FileSystemDocument, MemoryDocument y DocumentFromDocumentLibrary.


\subsection{Persistencia en Disco}
Respecto a la persistencia la implementamos utilizando un StraigthVariableLenthFile, teniendo la posibilidad de agregar un documento al archivo de documentos y recuperando su OffsetAdress; y a partir de este ultimo recuperar una instancia de una implementacion (DocumentFromDocumentLibrary) de Document persistida. Ademas de estos dos metodos existe uno para recuperar la cantidad de Document's persistidos, este valor es almacenado en el principio del archivo y es utilizado por el SearchEngine para resolver las consultas de busquedas.

\subsection{Document}
La interfase Documento es realmente simple, contiene un metodo para la apertura, cerrado, otro para determinar si se puede abrir y el metodo para obtener la linea actual, siendo iterable por este ultimo.

\subsection{Implementaciones de Document}

\paragraph{FileSystemDocument}
Las distintas implementaciones de la abstraccion de Document cumplen diferentes requerimientos segun su implementacion. La primera de ellas, FileSystemDocument, como indica su nombre permite recuperar las lineas de un archivo del fileSystem, siendo instanciada con su path, y es utilizado exclusivamente por el Main para abrir los documentos que el usuario desea agregar al sistema y reproducirlos.

\paragraph{MemoryDocument}
MemoryDocument permite tener un Document completamente independiente de su representacion en el fileSystem, simplemente se lo instancia y se le agregan lineas con su metodo addLine. Lo utilizamos para todo el manejo de documentos, entre esos usos, como criterio de busqueda al SearchEngine y recuperacion de un documento a partir de su OffSetAdress en el servicio de persistencia (DocumentLibrary).

\paragraph{DocumentFronDocumentLibrary}
DocumentFromDocumentLibrary es una capa de abstraccion que implementa Lazy Loading, es decir, hasta que realmente no se abra el documento el no lo buscara en DocumentLibrary.
  % Document, DocumentLibrary
\section{Crawler}
El crawler es una entidad bastante simple que se encarga de agregar documentos al sistema.

\subsection{bla}
          % Crawler, Discriminator
\section{Indexer}

\subsection{Indexer}

\subsection{Lexical Manager}
          % Indexer, LexicalManager
\section{BSharp Tree}

Durante toda esta secci�n se llamar� nodos internos a los nodos del denominado index-set y nodos hojas a los del sequential-set.

Por requerimientos era necesario tener la informaci�n de indexaci�n almacenada en un �rbol b\#. Se aclara que es por requerimientos puesto que bien podr�a haber sido un �rbol b* puesto que, al menos por esta entrega, no se hacen recorridos secuenciales en las hojas (aunque de todos modos la utilizaci�n de un �rbol b\# puede haber disminuido la cantidad de lecturas necesarias para acceder a un dato al menos \textit{en promedio}).

Se pens� entonces en tener un �rbol b\# gen�rico que fuera luego configurado de acuerdo a nuestras necesidades particulares de indexaci�n. Se observ�, en primer lugar, que el algoritmo general de un �rbol de este tipo no cambia (al menos no sustancialmente) si se tiene un �rbol b\# funcionando en memoria, sin ning�n tipo de persistencia, que si se tiene uno que funcione con su persistencia en disco.

Por ese motivo se decidi� programar un �rbol b\# abstracto con una implementaci�n en memoria y otra en disco.

NOTA: La implementaci�n en memoria no fue usada en el TP, pero el realizarla facilit� la programaci�n y testeo del algoritmo general del �rbol b\#. Se la incluye solo a tono informativo.

La complejidad de la creaci�n de una instancia de un �rbol queda oculta por la utilizaci�n de la clase \textit{BTreeSharpFactory} (que utiliza los patrones Facade y Factory).

Se describe en primer lugar la interfaz: 

\begin{figure}[!htp]
\centering
\makebox[\textwidth]{\includegraphics[scale=0.5,natwidth=20pt,natheight=10pt,width=1.15\textwidth]{img/BTree.png}}
\caption{Interfaz de un �rbol B}
\end{figure}

Los nodos internos poseer�n claves, representadas por la clase \textit{Key} cuya �nica particularidad a destacar es que deben ser comparables.

Los nodos hojas poseer�n \textit{Elementos} que tendr�n, obligadamente, una Key.

El �rbol posee m�todos para agregar un Elemento, para buscar una Key (obteniendo como resultado un Elemento) y para iterar las hojas (a partir de una cierta Key).

Si se agrega un Elemento cuya Key ya existe previamente en el �rbol entrar� en juego un m�todo definido por Elemento. Elemento define el m�todo \textit{updateElement()} que recibe un Elemento (que por contrato debe poseer su misma Key); el Elemento a ``updatear`` debe tomar el Elemento recibido y extraer de �l los datos necesarios para actualizarse. El resultado de esta operaci�n debe devolver un booleano indicando si la representaci�n del elemento para el �rbol fue modificada o no (De esta manera el �rbol puede saber si debe reescribir el n�do hoja correspondiente al elemento actualizado o no).

\subsection{BSharp Tree Abstracto}

El �rbol b\# abstracto implementa la interfaz de �rbol simplemente teniendo un nodo raiz dentro de �l, y delegando a este todas las operaciones.

Se posee una clase abstracta \textit{Node} que define m�todos similares a los de un �rbol b\# pero que, adem�s, pueden manejar el pasaje de Keys y de \textit{NodeReferences} entre nodos para resolver los casos en que exista overflow.

Una \textit{NodeReference} es una interface con un m�todo para obtener un nodo. Los nodos internos en lugar de tener dentro de si nodos, poseen NodeReferences que permiten obtener el nodo correspondiente. De acuerdo a la implementaci�n usada (disco o memoria) NodeReference buscar� el nodo correspondiente.

Existen cuatro implementaciones, tambi�n abstractas, de Node, que se corresponden con los diferentes tipos de nodos, a saber: raiz inicial, raiz definitiva, nodo interno, nodo hoja.

Debido a la necesidad de mantener los nodos con una capacidad m�nima ocupada de 2/3 de su tama�o, las raices deben ocupar el doble de lo que ocupa un nodo normal, para que as�, al producirse overflow en la raiz, los nodos producidos en el split resultante tengan 2/3 de su capacidad ocupada.

El nodo raiz inicial, representado por la clase abstracta \textit{AbstractEspecialRootNode}, posee dentro suyo los Elementos sin referencias a otros nodos (en este sentido se comporta parecido a un nodo hoja). Cuando se produce un overflow se divide en 3 nodos hojas, generando un nodo raiz definitiva.

El nodo raiz definitivo, representado por la clase abstracta \textit{AbstractRootNode}, posee dentro suyo las claves de indexaci�n necesarias y los NodeReferences correspondientes a esas claves de indexaci�n. Estos NodeReferences pueden apuntar a o bien nodos hojas o bien nodos internos. El nodo raiz recibe los resultados de splits o de ``pases`` entre hermanos producidos justo por debajo de �l. Esto puede producir un split de un nodo raiz (debido a un overflow) generando por resultado 3 nuevos nodos internos, y una reconfiguraci�n de la raiz. El comportamiento de la raiz definitiva es similar al de un nodo interno.

Como se dijo ambas raices, inicial y definitiva, tienen el doble de capacidad de un nodo normal; Adem�s, por motivos obvios, tienen relajada la condici�n de tener al menos 2/3 de su capacidad ocupada.

Los nodos internos, representados por la clase abstracta \textit{AbstractInternalNode}, poseen dentro suyo las claves de indexaci�n necesarias y los NodeReferences correspondientes a esas claves de indexaci�n. Estos NodeReferences pueden apuntar a o bien nodos hojas o bien nodos internos. El nodo interno recibe los resultados de splits o de ``pases`` entre hermanos producidos justo por debajo de �l. Esto puede producir o bien un split de un nodo interno, generando por resultado 3 nuevos nodos internos, o un pase de keys y referencias con un hermano. En ambos casos debe informarse al nodo superior (ya sea una raiz u otro nodo interno) del nuevo nodo generado (si hubo un split) y de la(s) nueva(s) clave(s) de indexaci�n. Un split se produce como la divisi�n en 3 partes de la uni�n de este nodo con un hermano.

Los nodos hojas, representado por la clase abstracta \textit{AbstractLeafNode}, poseen dentro suyo los elementos. El agregado de un elemento, ya sea por update de un elemento o por agregado real, puede producir el llenado del nodo, generando esto el ``pase`` de un Elemento a un hermano, o en caso de estar lleno tambi�n el hermano, un split de la uni�n de este y su hermano. En ambos casos debe informarse al nodo superior la misma informaci�n descripta en el nodo interno.

El mecanismo de pases y de split descripto en los nodos hojas es igual para los nodos internos.

\begin{figure}[!htp]
\centering
\makebox[\textwidth]{\includegraphics[scale=1.2,natwidth=20pt,natheight=10pt,width=1.15\textwidth]{img/BTreeSharpAbstracto.png}}
\caption{�rbol Abstracto}
\end{figure}

Todos los nodos descriptos dejan como abstractos ciertos m�todos que deben ser implementados luego (disco o memoria), pero que son invocados por la clase abstracta (patr�n template). Estos son:
\begin{enumerate}
\item \textit{postAddElement()}: Permite que la implementaci�n en disco se de por enterada de un cambio en un nodo permitiendo su grabaci�n.
\item \textit{calculateNodeSize()}: Calcula el tama�o de un nodo (para saber si se excedi� o no de su capacidad).
\item \textit{getParts()}: Permite dividir la uni�n de dos nodos (o uno solo en caso de raiz) en 3 partes. Posee particularidades seg�n se trate de nodos hoja/raizInicial o de nodos interno/raizDefinitiva.
\end{enumerate}

Adem�s se posee una interface factory, llamada \textit{BTreeSharpNodeFactory} de creaci�n de nodos para poder crear los nodos de la implementaci�n adecuada en el momento de producirse un split o cuando un NodeReference debe obtener el nodo. Esta interface debe ser implementada (disco o memoria) (patr�n AbstractFactory).

Por �ltimo todos los nodos (por estar esto en la clase Node) tienen acceso a las configuraciones (por ejemplo tama�o de los nodos) y a una referencia a si mismos (\textit{myNodeReference}).

\newpage

\subsubsection{Dise�o de datos}

Si bien ya fue adelantado en las secciones anteriores, este es el dise�o de datos (atributos de las clases correspondientes) por tipo de nodo:

\begin{verbatim}
NodeReference: referencia a nodo
KeyNodeReference: Key, NodeReference (union de ambos)
Nodo Raiz Inicial: NodeReference a si mismo, lista de elementos
Nodo Raiz Definitiva: NodeReference a si mismo,
           primera NodeReference, lista de KeyNodeReference
Nodo Interno: NodeReference a si mismo, primera NodeReference,
           lista de KeyNodeReference
Nodo Hoja: NodeReference a si mismo, NodeReference a nodo
           anterior, lista de elementos, NodeReference a 
           nodo siguiente
\end{verbatim}

\subsubsection{Configuraci�n de tama�o de los nodos}

Inicialmente se hab�a pensado en poder definir nodos de un tama�o en particular para los nodos internos y de otro tama�o para los nodos hojas. Pero se encontraron problemas que se detallan a continuaci�n.

Al tener overflow en la raiz original se generar�n 3 nodos hojas:

Si tienen igual tama�o:

$$size(Raiz) = 2*size(Interno) = 2*size(Hoja) ==> 1/3*size(Raiz) = 2/3*size(Hoja)$$

(Es decir, se cumple el invariante de que los nodos tienen al menos 2/3 de su capacidad llena)

Si el nodo interno es menor que el nodo hoja:

$$size(Raiz) = 2*size(Interno) < 2*size(Hoja) ==> 1/3*size(Raiz) < 2/3*size(Hoja)$$

(Es decir, NO se cumple el invariante de que los nodos tienen al menos 2/3 de su capacidad llena!! Al menos no se cumple ni bien se divide la raiz original. Luego si se cumplir�)

Si el nodo interno es mayor que el nodo hoja:

$$size(Raiz) = 2*size(Interno) > 2*size(Hoja) ==> 1/3*size(Raiz) > 2/3*size(Hoja)$$

(Si bien se cumple el invariante anterior, puede pasar que los nodos hojas creados queden inmediatamente en overflow! Esto se podr�a resolver con una nueva divisi�n inmediata de las hojas).

Si bien se puede relajar el invariante en la primera divisi�n (para el caso size(Interno)<size(Hoja)) e implementar una nueva divisi�n inmediata (para el caso size(Interno)>size(Hoja)), parece no valer la pena, por lo que se estableci� que ambos nodos, hoja e internos, tengan un mismo tama�o.

\subsection{Implementaci�n de BSharp Tree en Memoria}

NOTA: Esta implementaci�n no fue usada en el TP, pero el realizarla facilit� la programaci�n y testeo del algoritmo general del �rbol b\#. Se la incluye solo a tono informativo.

La implementaci�n en memoria implementa las referencias a nodos (NodeReference) conteniendo el nodo dentro de si mismo.

El m�todo \textit{calculateNodeSize()} calcula el tama�o del nodo contabilizando la cantidad de keys/elements del nodo (seg�n el tipo de nodo que se trate).

El m�todo \textit{getParts()} divide la uni�n de los dos hermanos (o la raiz si lo era) en tres partes que contengan la misma cantidad (o mejor aproximaci�n) de keys/elements.

El m�todo \textit{postAddElement()} no hace absolutamente nada.

\begin{figure}[!htp]
\centering
\makebox[\textwidth]{\includegraphics[scale=1.2,natwidth=20pt,natheight=10pt,width=1.15\textwidth]{img/BTreeSharpMemoria.png}}
\caption{Implementaci�n de �rbol en Memoria}
\end{figure}

\subsection{Implementaci�n de BSharp Tree en Disco}

Esta implementaci�n es bastante m�s extensa y complicada que la de memoria.

Por ser en disco era necesario decidir una estructura de archivos para poder albergar a los nodos. En primer lugar se decidi� que exista un archivo separado para los nodos hoja y otro para los nodos internos/raices. Adem�s se tom� la decisi�n de trabajar con archivos en bloques. Estos bloques deben coincidir con la capacidad m�xima de un nodo; estableci�ndose de esta manera una equivalencia bloque-nodo. De esta manera un nodo puede ser actualizado sin que se produzca un movimiento de lugar dentro del archivo (si esto sucediera habr�a que actualizar las referencias hacia ese bloque [nodo] lo cual, adem�s de ser engorroso, implicar�a lecturas y escrituras extras de disco lo cual har�a mucho m�s lento el �rbol).

La abstracci�n utilizada para manejar estos archivos ablocados y actualizables es \textit{VariableLengthFileManager}. Para mantener la correspondencia entre bloque y nodo, los serializadores correspondientes (de hojas, internos, etc. que parametrizan al VLFM) al momento de deshidratar un nodo rellenan los buffers con ``basura`` (relleno de 0s) hasta alcanzar la capacidad m�xima del nodo.

La interface NodeReference fue implementada de manera tal que posea el VLFM adecuado y la direcci�n a acceder (n�mero de bloque) dentro del archivo. El m�todo \textit{getNode()} obtiene el nodo en el momento de ser pedido (de manera lazy). Adem�s posee un m�todo \textit{saveNode()} que recibe como par�metro el nodo a grabar (algunos detalles adicionales luego).

El m�todo \textit{calculateNodeSize()} calcula el tama�o del nodo utilizando el m�todo \textit{getDehydrateSize()} del serializador correspondiente al nodo en cuesti�n (ver luego dise�o de datos/serializadores).

El m�todo \textit{getParts()} divide la uni�n de los dos hermanos (o la raiz si lo era) en tres partes usando un algoritmo m�s complicado que el usado en la implementaci�n en memoria. Se volver� a �l en la parte de Serializadores (a los que utiliza para realizar la divisi�n).

El m�todo \textit{postAddElement()} utiliza la referencia a si mismo (\textit{myNodeReference}), accediendo al m�todo \textit{saveNode()} y pas�ndose a si mismo (al nodo en si mismo) para que sea grabado. La implementaci�n de saveNode() si posee la direcci�n del nodo lo graba en ella; si no la posee agrega el nodo al VLFM y se guarda la direcci�n devuelta por este.

\begin{figure}[!htp]
\centering
\makebox[\textwidth]{\includegraphics[scale=1.2,natwidth=20pt,natheight=10pt,width=1.15\textwidth]{img/BTreeSharpDisk.png}}
\caption{Implementaci�n de �rbol en Disco}
\end{figure}

\newpage

\subsubsection{Dise�o de datos y Serializadores}

El dise�o conceptual de datos de acuerdo al tipo de nodo es el siguiente:

\begin{verbatim}
NodeReferenceDisk((direccion)1)
EspecialRootNodeDisk((elemento)*)
RootNodeDisk((tipo de referencias)1,(key)+,(referencia)+)
InternalNodeDisk((tipo de referencias)1,(key)+,(referencia)+)
LeafNode((referencia nodo anterior)?,(elemento)+,
        (referencia nodo siguiente))

\end{verbatim}

El dise�o conceptual de acuerdo al tipo de archivo es el siguiente:

\begin{verbatim}
Archivo de nodos internos:
  Bloques nodo raiz: NodoRaiz((tipo de nodo)1,
                     (((elemento)*)1 | 
                     ((tipo de referencias)1,(key)+,
                     (referencia)+)1)1)
  Resto de los Bloques: NodoInterno((tipo de referencias)1,
            (key)+, (referencia)+)

Archivo de nodos hojas:
NodoHoja((referencia nodo anterior)?,(elemento)+,
         (referencia nodo siguiente)?)
\end{verbatim}

Como puede verse en lo anterior, el archivo de nodos internos contiene 3 tipos diferentes de nodos: el nodos raiz, subdivido en dos tipos, el inicial y el definitivo, que tiene una ubicaci�n fija dentro del archivo, y el nodo interno.

El encargado de llevar a cabo el pasaje desde el objeto hacia su forma persistida y viceversa es el Serializador, existiendo uno para cada tipo de nodo. 

El �rbol en disco se parametriza, adem�s de con un tipo de Key y de Element, con un serializador de lista de Keys y un serializador de lista de Elements. Cada serializador de nodos delega la serializaci�n de sus Elements/Keys en estos serializadores de listas que parametrizan al �rbol.

Una alternativa al uso de serializadores de listas de Keys/Elements podr�a haber sido el uso de serializadores de Key / Element (de un solo objeto). Pero esto no hubiese permitido la aplicaci�n de frontcoding (porque la interfaz de Serializer no sabe lo que es un objeto anterior [y as� debe ser!]). Si bien en el uso final que se di� al �rbol en el TP no se hizo uso de frontcoding en los nodos internos (porque se ped�a que no fuera as�) el �rbol gen�rico que estamos explicando en esta secci�n lo soporta (si se pasa un serializador de lista de Keys adecuado).

\newpage

El dise�o l�gico de datos para los registros por tipo de nodo es:

\begin{verbatim}
EspecialRootNodeDisk((lista de elementos: tipo desconocido)1)
RootNodeDisk((tipo de referencias:Byte)1,
            (lista keys: tipo desconocido)1,(referencia:Long)+)
InternalNodeDisk((tipo de referencias:Byte)1,
            (tipo de referencias:Byte)1,(referencia:Long)+)
LeafNode((referencia nodo anterior:Long)1,
         (lista elements: tipo desconocido)1,
         (referencia nodo siguiente:Long)1)

\end{verbatim}

El dise�o l�gico de datos para los registros de acuerdo al tipo de archivo es el siguiente:

\begin{verbatim}
Archivo de nodos internos:
  Bloques nodo raiz: 
    NodoRaiz((tipo de nodo:Byte)1, (((lista de
      elementos: tipo desconocido)1)1 | ((tipo de 
      referencias:Byte)1, (lista keys: tipo desconocido)1,
      (referencia:Long)+)1)
  Resto de los Bloques: 
    NodoInterno((tipo de referencias:Byte)1,
                (tipo de referencias:Byte)1,
                (referencia:Long)+)

Archivo de nodos hojas:
  NodoHoja((referencia nodo anterior:Long)1,
           (lista elements: tipo desconocido)1,
           (referencia nodo siguiente:Long)1)
\end{verbatim}

En lo anterior cuando se dice ``tipo desconocido`` se lo hace porque el dise�o de los datos est� a cargo de los serializadores de Element/Key que parametrizan al �rbol. Adem�s, se hace notar que en el caso de RootNodeDisk e InternalNodeDisk no es necesario incluir la cantidad de referencias puesto que ser�n una m�s que la lista de keys (que, estrat�gicamente fue puesta antes que la lista de referencias).

Conocida la forma de serializar los nodos puede volverse al problema de la divisi�n en tercios dejado inconcluso anteriormente. El algoritmo divide en tercios la lista de keys/elements utilizando auxiliarmente el m�todo \textit{getDehydrateSize()} de los serializadores correspondientes. En primer lugar se hace la divisi�n en tres partes usando como criterio la cantidad de objetos contenida por la lista. Luego se miden los tama�os de las partes usando el m�todo \textit{getDehydrateSize()} y se pasan elementos entre las partes hasta obtener la mejor aproximaci�n (por ser elementos de longitud variable es altamente probable que no se consigan tercios exactos) al tercio buscado, remidiendo en cada pasaje el tama�o de los tercios (esto es necesario en cada paso porque no se sabe de que manera serializa el serializador que parametriza al �rbol [puede, por ejemplo, estar aplicando frontcoding]). Una vez obtenido los diferentes tercios de la lista de keys/elements, se arman los tres nodos resultantes.
       % B# Tree
\section{SearchEngine: Queries y FTRS}

\subsection{bla}
     % SearchEngine, Queries, FTRS
\section{Diccionario de Palabras: Trie}
Para esta entrega, las palabras del diccionario fueron almacenadas en un Trie de profundidad parametrizable (por defecto 4). 
Esta estructura presenta una cualidad importante en nuestro contexto que es la optimizaci�n de las busquedas de claves. 
La b�squeda de una clave de longitud $n$ tendr� en el peor de los casos un orden de $O(m)$.

No se explicar�n aqu� todos los detalles espec�ficos de la estructura del Trie, sino que se har� referencia a alguno de
los detalles conceptualmente m�s importantes de la implementaci�n para este trabajo. \footnote{Puede encontrarse m�s
informaci�n en http://es.wikipedia.org/wiki/Trie}

\subsection{Implementaci�n en Disco}
A pesar de la estructura relativamente simple del Trie, este ocupa un tama�o considerable cuando se trata de almacenarlo
en disco por completo. La estructura en s� no impone ninguna limitaci�n sobre la cantidad de niveles que deba tener el
Trie, y este es un factor importante a tener en cuenta cuando la implementaci�n se hace en disco.

\paragraph{Profundidad del Trie}
Limitar la profundidad de niveles reduce el espacio necesario para almacenar el Trie, y generalmente es un requisito
de la implementaci�n en disco. Como consecuencia de la limitaci�n de niveles, el Trie pierde (en parte) su gran ventaja
que es la rapidez de las busquedas de claves. Esto puede verse claramente pensando que, al buscar una clave de longitud
mayor a N (siendo N la cantidad de niveles), solo se podr� bajar hasta el �ltimo nivel y luego se deber� hacer una 
busqueda secuencial (o binaria si se mantienen ordenadas) sobre el resto de las claves que compartan los mismas N primeros caracteres.

Considerando este problema, debemos encontrar una soluci�n de compromiso entre el espacio necesario para almacenar el
Trie y la rapidez de la busqueda de clavez que proporcionar�. Usualmente, limitando la cantidad de niveles a un valor 
entre 4 y 6, se logra una soluci�n bastante equilibrada.

\paragraph{Nodos Hojas: Cantidad de registros}
Otro problema que surge de la limitaci�n de la cantidad de niveles del Trie es la cantidad de registros que se necesitan
almacenar en los nodos hojas.

Debido a que muchas claves pueden tener en comun los primero caracteres, algunos nodos hojas pueden tener una cantidad
elevada de registros con terminaciones de claves, lo cual limita la posibilidad de levantar el nodo completo en memoria.
Para este problema se dise�o una soluci�n que explicaremos m�s adelante.

\subsection{Dise�o del Trie gen�rico}
Se intent� realizar una implementaci�n gen�rica del Trie. Para logra este objetivo, fue necesario definir interfaces 
extras que permitan trabajar con distintos tipos de datos.

\paragraph{Definici�n de Entidades}

Las entidades b�sicas son Element, Key y KeyAtom:

\begin{verbatim}
Elemento (Element):
  Es el elemento (o dato) que almacena el Trie.

Clave (Key):
  Es la clave mediante la cual se recorre el Trie. No necesariamente 
  debe ser una cadena de caract�res.

Porci�n de Clave (KeyAtom):
  Cada clave se divide en porciones ordenadas que se almacenan en 
  cada nivel. No necesariamente tiene que ser un caracter ya que 
  la clave podr�a dividirse en m�s de un caracteres. 
\end{verbatim}

Luego las entidades que le dan al Trie su estructura, InternalNode, LeafNode y NodeReference:

\begin{verbatim}
Referencia a Nodo (NodeReference): 
  Representa una referencia a un nodo (nodo hijo). Contiene 
  la direcci�n del nodo y la porci�n de clave del nodo hijo.
  
Nodo Interno (InternalNode):      
  Representa un nodo que no es hoja. Puede tener un dato si se
  agrega una clave que sea mas corta que la cantidad de niveles.
  
Nodo Hoja (LeafNode):
  Este nodo puede contener varios elementos ya que puede haber
  varios elementos cuya clave tengan el mismo comienzo. 
\end{verbatim}

\paragraph{Nodos}
Si no fuera por la limitaci�n de niveles, muy probablemente no ser�a necesario hacer una diferenciaci�n entre nodo
interno y nodo hoja. Sin embargo, en nuestro caso, los nodos hojas \textbf{nunca} contienen referencia a nodos hijos. 

\begin{figure}[!htp]
	\begin{center}
		\includegraphics[scale=0.5,natwidth=20pt,natheight=10pt]{img/trie_simple.png}
	\end{center}
	\caption{Relaciones entre Nodos (Ej: Mariana,Marcos,Mario,Mabel)}
\end{figure}

\newpage

De la figura se puede prever que un nodo hoja podr�a eventualmente tener muchas terminaciones de claves (Mariana, Marcos,
Marcelo, Martes, Marciano, Mariano, Mario, etc). Si suponemos un numero considerable como 50 o 100 elementos en un nodo 
hoja, ya no podemos considerar que el nodo este por completo en memoria. 

Para solucionar este problema, se decidi� tratar a los nodos hojas con \textbf{particiones}. En vez de tener una entidad que 
represente un nodo hoja, se tiene una entidad que representa una parte de ese nodo hoja. De esta manera podemos levantar
a memoria, partes del nodo hoja y no el nodo completo.

Definimos entonces el concepto de \textbf{Partici�n de Nodo Hoja} \textbf{(LeafPartitionNode)}, el cual puede contener 
adentro una cantidad fija de registros. \footnote{Actualmente utilizamos una cantidad de 10 registros por partici�n.}

\begin{figure}[!htp]
	\begin{center}
		\includegraphics[scale=0.5,natwidth=20pt,natheight=10pt]{img/trie_particion.png}
	\end{center}
	\caption{Relaciones entre Nodos Internos y Nodos Hojas} 
\end{figure}

\paragraph{Clave y Elemento}
Se implementaron clases particulares para la clave y el elemento. Para la clave se utilizo un StringKey, y en nuestro 
caso el elemento es el offset en el archivo de audio. Las KeyAtom, en nuestro caso son caracteres individuales.
Con lo cual donde corresponda \textbf{KeyAtom} se usar� un \textbf{char} y en donde corresponda \textbf{Element} se 
usar� un \textbf{long}.

\subsection{Dise�o de Datos}
Teniendo en claro las entidades involucradas en la estructura del Trie, se realiz� un dise�o de los datos que ser�an
necesarios almacenar.

\begin{verbatim}
Nodo Interno: nivel, elemento (o dato), lista de referencias
Referencia a Nodo: porci�n de clave, lista de direcciones
Particion de Nodo Hoja: resto de la clave, elemento (o dato)
\end{verbatim}

Se implementaron dos archivos, ambos con \textbf{registros de longitud variable en bloques}, uno para los nodos registros
de los nodos internos y otro para los de nodos hojas.

\paragraph{Dise�o Conceptual de Datos}
Se realiz� un dise�o conceptual de datos para los registros:
\begin{verbatim}
Arhivo de nodos internos:
NodoInterno((nivel)1, (dato)?, referencia a nodo(porcion clave,
direccion((nro bloque)1, (nro objeto)1)*1 )*)

Archivo de nodos hojas:
ParticionNodoHoja( (resto clave((porcion clave)*))1, (dato)1 )
\end{verbatim}

\paragraph{Dise�o L�gico de Datos}
Se realiz� un dise�o l�gico de datos para los registros:
\begin{verbatim}
Arhivo de nodos internos:
NodoInterno(nivel:Short, dato:Elemento, (porcion clave:KeyAtom, 
direccion(nro bloque:Long, nro objeto:Short)))

Archivo de nodos hojas:
ParticionNodoHoja(Cant de KeyAtom en resto de clave:Short, 
(porcion clave:KeyAtom), dato:Long)
\end{verbatim}

             % Diccionario, Trie

\end{document}
