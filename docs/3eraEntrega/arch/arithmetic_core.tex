\section{Procesado Aritmético}
Todo el manejo de rangos propio del aritmético se concentró en la clase \textit{ArithmeticProcessor}. La cual mantiene el estado de piso y techo y, además, conoce cuando deben generarse overflows y underflows. Cada vez que se le pide procesa un único paso de aritmético (sin importar si es compresión o descompresión) con una tabla que se le pasa por parámetro y un objeto que se encarga únicamente de decirle cual es la posición de la tabla de probabilidades que tiene que utilizar para quedarse como nuevo techo y piso (a este objeto se lo denomina matcher).
Esta clase es abstracta para que sus subclasificaciones definan la acción a realizar (comprimir o descomprimir) ya que no define ninguna interfaz pública. Las subclases dirán que se hace con el resultado del proceso de la tabla. Este procesado invoca métodos templates para hacerle saber a sus descendientes que está ocurriendo en el process (por ejemplo, avisa cuando ocurren overflows y de que bits, etc.). 
Por ejemplo, para el caso de compresión, el objeto de matcher definido se basa en el caracter a comprimir y cada vez que ocurre overflow o se "limpian" los underflows acumulados emite el bit correspondiente.
En cambio, el caso de descompresión, toma de una entrada los bits y el matcher se basa en que valores tiene el rango para decidir en que momento debe para el proceso de la tabla. Cuando ocurren los overflow o underflows el mismo descarta de su valor actual dichos bits y solicita mas bits a la entrada. También verifica que los bits ocultados cuando ocurrió la el underflow sean los opuestos al primer overflow que ocurra.
Esto también nos facilitó el agregado de \textit{tracers} para el aritmético de Orden 1 que se implementó para la prueba de este módulo.
\subsection{Compresor y Descompresor Aritmético}
Como se comentó anteriormente, ambas clases extienden del proceso aritmético. Tratan de a un caracter por vez para que los compresores que se utilizan dentro de su forma de trabajo un aritmético puedan usarlo sin problema y nunca modifican la tabla de probabilidades que reciben ya que los usuarios de estas clases son los encargados de hacer mantenimiento de las mismas.
El Compresor, como se mencionó anteriormente, sólo emite los overflow que el proceso le indique. El matcher que define se basa en el caracter a comprimir. Mientras que el descompresor toma bits de la entrada cada vez que necesita nueva información (esto ocurre cuando se detectan overflows o underflows).
