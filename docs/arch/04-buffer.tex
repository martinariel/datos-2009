\section{Buffers de Entrada y Salida}

El m�dulo de Buffers es quiz�s el mas sencillo de todos. Los buffers funcionan como una memoria de almacenamiento 
temporal de la informaci�n. Normalmente, se hace uso del buffer como una memoria intermedia, �til para almacenar 
informaci�n que est� por escribirse en un archivo o que haya sido le�do de uno.

Los buffers suelen mejorar el rendimiento en el intercambio de datos entre diferentes m�dulos de la aplicaci�n.

\subsection{An�lisis}

Los requisitos son claros: se necesita un Buffer de Entrada, del cual se leer�n los datos cargados del archivo, y un 
Buffer de Salida que ser� utilizado para escribir los datos que necesiten ser persistidos:

\begin{enumerate}
  \item \textsf{Buffer de Entrada}: Debe poder ser cargado con los datos que se lean de los archivos y permitir� 
    recuperar dichos datos para ser utilizados en la aplicaci�n.
  \item \textsf{Buffer de Salida}: Debe poder ser cargado con los datos que necesiten ser persistidos 
\end{enumerate}

\subsection{Soluci�n Propuesta}

texto
