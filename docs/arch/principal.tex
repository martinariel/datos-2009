%%%%%%%%%%%%%%%%%%%%%%%%%%%%%%%%%%%%%%%%%%%%%%%%%%%%%%%%%%%%%%%%%%%%%%%%%%%%%%%
% Definici�n del tipo de documento.                                           %
% Posibles tipos de papel: a4paper, letterpaper, legalpapper                  %
% Posibles tama�os de letra: 10pt, 11pt, 12pt                                 %
% Posibles clases de documentos: article, report, book, slides                %
%%%%%%%%%%%%%%%%%%%%%%%%%%%%%%%%%%%%%%%%%%%%%%%%%%%%%%%%%%%%%%%%%%%%%%%%%%%%%%%
\documentclass[11pt, spanish, a4paper]{article}

%%%%%%%%%%%%%%%%%%%%%%%%%%%%%%%%%%%%%%%%%%%%%%%%%%%%%%%%%%%%%%%%%%%%%%%%%%%%%%%
% Los paquetes permiten ampliar las capacidades de LaTeX.                     %
%%%%%%%%%%%%%%%%%%%%%%%%%%%%%%%%%%%%%%%%%%%%%%%%%%%%%%%%%%%%%%%%%%%%%%%%%%%%%%%
\usepackage[spanish]{babel}     % Paquete para definir el idioma usado.
\usepackage[latin1]{inputenc}   % Define la codificaci�n de caracteres 
                                % (latin1 es ISO 8859-1)
%\usepackage[T1]{fontenc}        % Agrega caracteres extendidos al font
\usepackage{t1enc}
\usepackage{palatino}           % Cambia el font por omision a Palatino
\usepackage{graphicx}           % Paquete para inclusi�n de gr�ficos.
%%%%%%%%%%%%%%%%%%%%%%%%%%%%%%%%%%%%%%%%%%%%%%%%%%%%%%%%%%%%%%%%%%%%%%%%%%%%%%%

% T�tulo principal del documento.
\title{\textbf{The Speaker (Trabajo Pr�ctico Nro. 1)}}

% Informaci�n sobre los autores.
\author{    
            Juan Manuel Barrenche, \textit{Padr�n Nro. 86.152}               \\
            \texttt{ snipperme@gmail.com }                                   \\
            Mart�n Fern�ndez, \textit{Padr�n Nro. 88.171}                    \\
            \texttt{ tinchof@gmail.com }                                     \\
            R�l Lopez, \textit{Padr�n Nro. 88.430}                           \\
            \texttt{ rau\_carpo@hotmail.com }                                \\
            Marcos J. Medrano, \textit{Padr�n Nro. 86.729}                   \\
            \texttt{ marcosmedrano0@gmail.com }                              \\
            Federico Valido, \textit{Padr�n Nro. 82.490}                     \\
            \texttt{ fvalido@gmail.com }                                     \\ 
                                                                             \\
            \normalsize{Grupo Nro. 11 (YES) - 1er. Cuatrimestre de 2009}     \\
            \normalsize{75.06 Organizaci�n de Datos}                         \\
            \normalsize{Facultad de Ingenier�a, Universidad de Buenos Aires} \\
       }
\date{DIA de MES de 2009}


% define que archivos seran realmente incluidos
%\includeonly{intro,arch,file,buffer,serializer,persistence}

% Comienzo del documento
\begin{document}

\maketitle                % Inserta el t�tulo.
%\thispagestyle{empty}     % Quita el n�mero en la primer p�gina.

% Resumen que aparece en la primera p�gina (antes de la tabla de contenidos)
\begin{abstract}
Este parrafo deber�a contener una breve descripci�n del trabajo practico. No 
deber�a contener mas de 100 palabras.
En una introducci�n posterior se podr� describir con m�s detalle los contenidos
de este trabajo.
Documento desarrollado en LATEX.
\end{abstract}

\tableofcontents        % Comentar si no se desea la Table Of Content.


% Inclusi�n de archivos externos
\input{a}
%\include{intro}         % Introducci�n
%\section{Arquitectura}

El Main ser� simplemente una vista de WordService que es el backend de la aplicaci�n y quien brinda todos los servicios a la vista. Estos servicios son 3:
\begin{enumerate}
\item Agregar un Documento.
\item Realizar una b�squeda.
\item Reproducir un Documento.
\end{enumerate}

Veamos la arquitectura parte por parte. Pero antes definamos el concepto de Documento, que es com�n a todas las partes.

\subsection{Documento}

Es una interfaz que permite manejar un documento obteniendo de �l renglones.
Posee tres implementaciones, una que contiene un documento totalmente en memoria, otra que trabaja con un archivo que lee desde disco a medida que va necesit�ndolo, y otra que trabaja con los documentos prealmacenados que obtiene de forma \textit{lazy}.

Ahora si, veamos la arquitectura de cada una de las partes antes mencionadas.

\subsection{-1- Agregar un Documento}

Se divide en divide en 2 grandes partes:
\renewcommand{\labelenumi}{\alph{enumi}.}
\begin{enumerate}
\item Agregarlo al indice.
\item Agregarlo al diccionario de sonidos.
\end{enumerate}

Existe una caracter�stica en com�n entre estas dos partes: el parseo. Es por eso que la arquitectura se defini� de tal forma que el parseo se haga en una sola oportunidad como se mostrar� luego (esto tiene una peque�a consecuencia que se mencionar� al explicar la adici�n al diccionario de sonidos).

\subsubsection{-a- Agregar un Documento al indice}


Quien se encarga de realizar esto es el \textit{Crawler}, quien recibir� un \textit{Documento}. En primer lugar le pasa el documento al \textit{Parser} que lo utilizar� como fuente para datos.

El Parser permite manejar un documento en forma de \textit{Frases}, entendiendo por tal a un conjunto de palabras separadas por un signo de puntuaci�n (\textit{'.', ',', ';', '?', '!'}, etc.). A estas frases les aplicar� un reemplazo de s�mbolos diacr�ticos, retiro de s�mbolos extra�os, s�mbolos num�ricos, etc, y case folding, obteniendo por resultado una lista de palabras limpia. De esta manera el \textit{Crawler} puede obtener \textit{Frases} limpias desde el parser al que inicialmente le pasa el documento como fuente.

La colecci�n de palabras obtenida ser� procesada por el m�dulo \textit{StopWords Discriminator}, cuya implementaci�n tendr� un listado en memoria (que levanta al iniciar la aplicaci�n desde un archivo) de frases y de palabras sueltas que son stop words para hacer el procesamiento.

El \textit{Crawler} obtiene del \textit{StopWords Discriminator} una colecci�n (ordenada o desordenada, no importa, pero con repeticiones, si importa) de palabras que no contienen stop words. Esta colecci�n (correspondiente a una sola \textit{Frase}) le ser� entregada al \textit{Indexer}.

El \textit{Indexer} es la interfaz utilizada para acceder al �ndice (un \textit{�rbol b\#}) (tambi�n podr�a haberselo llamado Index en vez de Indexer). El \textit{Crawler} maneja una session con el \textit{Indexer}, que comienza al procesar un \textit{Documento} y termina al agregar la �ltima \textit{Frase} (cuando el \textit{Parser} deja de brindarle datos). El \textit{Indexer} parametriza un \textit{BTree\#}, un m�dulo que maneja un �rbol b\# de manera gen�rica. El \textit{Indexer} a su vez maneja una session id�ntica a la que tiene con el \textit{Crawler} con el m�dulo \textit{Inversion Sort Handler}, y solo agregar� las palabras al �rbol una vez que el \textit{Crawler} finalice la session con el \textit{Indexer}.

El \textit{BTree\#} es una implementaci�n gen�rica de �rbol b\# que utiliza 2 archivos, uno para nodos internos y otro para hojas, y que maneja una equiparaci�n entre nodos y bloques. Utiliza el \textit{VariableLengthFileManager} en su implementaci�n en disco. Debe ser parametrizado con una \textit{Key} (clave de los nodos internos) y un \textit{Element} (registro de las hojas que contiene entre otras cosas a la clave). Este \textit{Element} se definir� de tal manera que tenga el listado de \textit{Documentos} en su interior; este listado de \textit{Documentos} se implementar� utilizado el \textit{VariableLengthFileManager}. No se muestra a \textit{Key} ni a \textit{Element}, ni al archivo del listado de \textit{Documentos} en el diagrama por simplificaci�n. Adem�s deben brindarse los \textit{Serializer} correspondientes a una colecci�n ordenada de \textit{Keys} y de \textit{Elements} para parametrizar al \textit{BTree\#}.

Las sessiones mencionadas (entre el \textit{Crawler} y \textit{Indexer}, y entre el \textit{Indexer} y el \textit{Inversion Sort Handler}) sirven para evitar tener un documento completo en memoria.

Al mismo tiempo que el \textit{Crawler} va procesando las frases, va a agregando todas las palabras encontradas en un Set (conjunto sin repeticiones), de esta manera obtiene el vocabulario completo. Ese Set ser� devuelto al \textit{WordService} como respuesta al pedido de indexar un documento.

\subsubsection{-b- Agregar un Documento al diccionario de sonidos =-}


Como puede adivinarse, el vocabulario completo obtenido al agregar un docuemento al �ndice es la entrada usada para realizar el agregado al diccionario de sonidos. Esto tiene como consecuencia que el agregado de sonidos ser� en distinto orden al original del documento. Pero tiene como ventaja el no tener que procesar dos veces el documento por el \textit{Parser} (y a no tener que leerlo de nuevo desde disco si es que no se lo quiere tener completo en memoria como planteamos antes).

El m�dulo usado para esta acci�n es el llamado \textit{WordsRecorder}.

Si bien se modific� levemente la interacci�n entre los m�dulos, la arquitectura es b�sicamente la misma que para la entrega 1, y la interacci�n entre m�dulos puede verse claramente en el esquema.

La �nica gran modificaci�n consiste en el reemplazo del "�ndice secuencial de sonidos" por un \textit{Trie}, que utiliza para su implementaci�n un \textit{VariableLengthFileManager}.

\subsection{-2- Realizar una B�squeda}

Esta acci�n es responsabilidad del \textit{Search Engine}. Para hacerlo primero debe limpiar la cadena de b�squeda, para lo cual interactura con el \textit{Parser} y con el \textit{StopWords Discriminator}.
Una vez que tiene la consulta limpia, realiza la b�squeda utilizando el \textit{Indexer}, que a su vez utilizar� el \textit{BTree\#}.

El \textit{SearchEngine} realiza las operaciones de ordenado necesarias y devuelve un listado ordenado de \textit{documentos}.

\subsection{-3- Reproducir un Documento}

La reproducci�n la realizar� el \textit{DocumentPlayer}, que recibir� un \textit{Documento} que deber� parsear utilizando el \textit{Parser}. A medida que va obteniendo las palabras correspondientes las ir� reproduciendo delegando toda la complejidad en el \textit{WordsPlayer}.

El \textit{WordsPlayer} es la contrapartida del \textit{WordsRecorder} explicado antes, y su arquitectura es coincidente con la de este.

\subsection{Otros datos relevantes de la arquitectura}

Se defini� una abstracci�n de manejo de registros de longitud variable llamada \textit{StraightVariableLengthFile}. La misma solo permite agregado, sin actualizaciones. Se incorpor� esta abstracci�n al archivo de sonidos (que antes era un \textit{VariableLengthFileManager}) por lo que ahora el archivo correspondiente es 100\% denso. Adem�s fue usado en otras partes que lo requer�an como puede verse en el diagrama.

El \textit{VariableLengthFileManager} que ya exist�a en la entrega anterior, ahora permite actualizaciones de los registros.

\begin{figure}[!htp]
	\begin{center}
		\includegraphics[scale=0.5,natwidth=20pt,natheight=10pt]{img/Arch2daEntrega.png}
	\end{center}
	\caption{Arquitectura VariableLengthFileManager} 
\end{figure}

\begin{figure}[!htp]
	\begin{center}
		\includegraphics[scale=0.5,natwidth=20pt,natheight=10pt]{img/VariableLengthFileManager.png}
	\end{center}
	\caption{Arquitectura VariableLengthFileManager} 
\end{figure}
          % Descripci�n general de la Arquitectura e introducci�n de cada Modulo
%\include{file}          % Manejadores de Archivos
%\include{buffer}        % Buffers de Entrada y Salida
%\include{serializer}    % Serializadores
%\include{persistence}   % Servicios de Persistencia

\end{document}
