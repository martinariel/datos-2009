\section{Servicios de Documento (Backend)}
Dentro de este paquete se encuentra la clase principal de interaccion con el sistema, basicamente es lo unico que una View conoceria.

\subsection{WordService}

\paragraph{Interacciones}
Se encarga de instanciar el Indexer, Crawler, SoundPersistenceService utilizando un Trie, DocumentLibrary, SearchEngine, StopWordsDiscriminator y orquestar las relaciones entre ellos.

\paragraph{Metodos}
Expone los metodos principales para agregar, buscar y reproducir documentos. Que son delegados a los demas objetos.

\paragraph{Conexion con una View}
La conexion con la View se hace explicitamente en los metodos par agregar y reproducir a traves de una interfase (IWordsRecordConnector), de esta manera el servicio no conoce especificamente que view lo utiliza.

\subsection{Crawler}
El crawler es una entidad bastante simple que se encarga de agregar documentos al sistema.

\subsection{bla}

