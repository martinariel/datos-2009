\section{Introducci�n}

En las siguientes secciones trataremos de explicar el funcionamiento integral de nuestra soluci�n para el sistema {\it ** The Speaker **}.

El problema planteado, para esta primera entrega, consta de dos partes o funcionalidades principales. El sistema debe ser capaz de cargar textos, durante esta carga, cada termino dentro del texto que no exista debe permitirle al usuario grabar el audio asociado al mismo. Es decir, que se registrar� esa palabra y se la asociar� al audio capturado por el micr�fono del usuario y que el mismo confirme. La otra funcionalidad importante es la de reproducci�n de un archivo completo. Esta opci�n recorrer� todos los terminos del archivo que el usuario elija y reproducir� su audio asociado. Los terminos desconocidos no ser�n tenidos en cuenta para la reproducci�n del audio.

La soluci�n fue implementada en Java. Intentando concentrarse en dejar la suficiente flexibilidad en el sistema para aplicar modificaciones para las siguientes entregas que requieren la ampliaci�n de funcionalidad en el programa y optimizaci�n de la recuperaci�n de la informaci�n manejada. Para lograr esto se realiz� una divisi�n del sistema en m�dulos, como se puede ver en la definici�n de la arquitectura, con responsabilidades claramente delimitadas.

La divisi�n de trabajos fue en base a estos m�dulos (descritos en la secci�n de arquitectura). Los diferentes m�dulos fueron implementados por diferentes miembros del grupo en colaboraci�n con los responsables de los m�dulos que tuvieran interacci�n con el modulo en cuesti�n. En una reuni�n inicial se bosquejaron las ideas para la soluci�n. Que luego fueron separadas en m�dulos y asignados a cada uno de los integrantes. 

Para el manejo de la informaci�n que administra  {\it ** The Speaker **} y la organizaci�n de los datos nos basamos en lo explicado durante las clases te�ricas dictadas en la materia.

La documentaci�n, exceptuando el manual del usuario, fue creado con \LaTeX.

\footnotetext[1]{Para la documentaci�n del usuario se recomienda leer el documento {\it user.pdf} que se encuentra en la ruta docs/user/}

