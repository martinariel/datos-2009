\section{Compresión PPMC}

Como en los demás compresores implementados, se realizó una separación entre el compresor y el descompresor. 
Las clases y las interfaces son relativamente simples. Existe sin embargo funcionalidad exactamente igual en ambos, 
con lo cual algunas cosas se generalizaron.

\subsection{Compresor PPMC}
Se crea un compresor indicando el orden a utilizar (hasta 4) y opcionalmente puede recibir un PrintStream en el que irá
logueando información de debugging que puede ser útil (que se está emitiendo en cada caso, y como quedan los contextos).
Para comprimir un documento, se le debe pasar al compresor un \textit{Document} y un \textit{OutputBuffer} en el que irá
guardando el resultado de la descompresión. 
Al igual que en el LZP, el buffer es utilizado para crear un compresor aritmético. El aritmetico va comprimiendo los
caracteres que el PPMC le vaya pasando, junto con la tabla de probabilidad del contexto adecuado (de esto último 
obviamente se encarga el Compresor PPMC). El aritmético envía los bits correspondientes a la compresión al buffer.

\subsection{Descompresor PPMC}
El descompresor se crea de la una manera identica al compresor, es decir, se le puede pasar el orden a utilizar y un
PrintStream.
Para la descompresión se le debe pasar un \textit{InputBuffer} que contenga el mensaje a descomprimir. Se utiliza un
Aritmético que interpreta los bits segun las tablas de probabilidad que le pase el descompresor PPMC. El aritmético lee
los bits desde el InputBuffer recibido.
Después de realizar la descompresión de todos los caracteres (y encontrar un \textit{End-Of-File}) se devuelve un String
que representa el mensaje original.

\subsection{Contextos}
Tanto el compresor como el descompresor utilizan una estructura idéntica de contextos para mantener un estado coherente.
La forma de implementar esto, como se mostró en la arquitectura propuesta, fue la de crear una estructura de contextos 
encadenados.
El compresor/descompresor PPMC tienen a su alcance los contextos de orden 0 y de orden -1. A partir del orden 0, se 
puede encontrar cualquier contexto utilizando una List que indique el contexto buscado (Por ejemplo puede contener 
[A,B,C] y se buscara el contexto correspondiente a "ABC").
Cada contexto tiene una tabla de probabilidad que registra las ocurrencias de caracteres para ese contexto. Se pueden
agregar ocurrencias de caracteres al contexto, que repercutiran en agregar ocurrencias de caracteres en la tabla.

\subsubsection{Tipos de Contextos}
Se crearon 2 tipos de contextos, un contexto base con una implementación básica y un contexto que permite encadenar más
contextos. Si bien no difieren demasiado entre sí, tienen la diferencia importante de poder o no encadenar contextos.
El contexto base es utilizado en el orden -1 y en el ultimo orden posible. El otro tipo de contexto es utilizado en los
ordenes 0 a n-1.
Este último permite que al agregar una ocurrencia de un caracter al contexto, si el caracter no existía previamente en
dicho contexto, se crea un nuevo contexto del tipo adecuado, y se linkea contra este contexto.

\subsection{Algoritmo de Compresión y Descompresión}
El algoritmo utilizado para la compresión y la descompresión es bastante similar en esencia. Se tiene una representación
en caracteres del contexto actual y se utiliza para saber que contextos será necesario acceder. 
Se parte del contexto de orden 0 y se recorren los contextos recursivamente. Generalmente se llega al último contexto 
(de orden n) y se realiza el proceso de emisión recorriendo los contextos, finalizada está parte se hacen 
actualizaciones en las tablas si es necesario. 
El contexto de orden -1 es generalmente tratado de manera distinta, ya que aquí no hay que agregar ocurrencias de 
caracteres sino eliminar caracteres una vez que fueron emitidos.

\subsubsection{Exclusión completa}
Para aplicar exclusión completa simplemente se crea un \textit{Set} (\textit{HashSet}) al iniciar la descompresión y se
van agregando los caracteres que necesitan excluirse. De esta manera cada vez que se va a emitir algo, se aplica antes
a la tabla de probabilidades este set de exclusión.

\subsection{Serializer}

Para integrar el compresor con nuestra arquitectura, fue necesario crear un \textit{PPMCSerializer} que utiliza el 
compresor y descompresor PPMC para hidratar y deshidratar Documentos.
