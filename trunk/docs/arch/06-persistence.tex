\section{Servicios de Persistencia}

La persistencia ten�a como requerimiento t�cnico que los datos que maneja **The Speaker** fueran persistidos en dos archivos separados. En esta primera entrega el requerimiento era que fueran estrictamente dos archivos con la siguiente estructura l�gica definida.
\newcounter{archivo}
\begin{list}{\addtocounter{archivo}{1}\arabic{archivo}}
{
\setlength{\rightmargin}{\leftmargin}
}
\item \texttt{((palabra)i, offset)}
\item \texttt{(stream de audio)}
\end{list}
El primero est� compuesto de la dupla palabra (que es un identificador) y offset/referencia. Simbolizando que para la {\it palabra} el audio se encuentra en {\it referencia}.
El segundo archivo contiene los streams de audio capturados.

\subsection{An�lisis}
Lo primero que se observa de ambos archivos es que sus registros son de longitud variable y que hay homogeneidad en los datos a almacenar, es decir, que en cada archivo se almacenan siempre los mismos tipos de datos. De manera que, ambos archivos, a pesar de tener naturalezas de datos diferentes requieren el mismo manejo. Por lo cual, admiten una misma soluci�n de manejo del archivo mientras que la misma se mantenga indepen de la naturaleza de los datos a almacenar. 

Por otra parte, las operaciones que se deben permitir son el agregado de registros y la consulta de los mismos. El agregado de registros no requiere, en ninguno de los dos casos, que se haga con orden alguno. Mientras que la recuperaci�n de los datos, por otra parte, en el primer archivo debe poder ser secuencial (ya que se necesitan poder acceder a cada una de las palabras) y en el segundo caso debe poder accederse directamente a un registro que conozco su posici�n dentro del archivo. Por lo cual, estamos ante una organizaci�n secuencial de acceso relativo. Pudiendo recorrerse tanto secuencialmente como acceder a un registro espec�fico (si se conoce previamente su direcci�n). La implementaci�n de esta funcionalidad se hizo en la clase  {\textbf{VariableLengthFileManager}.}

\subsection{Soluci�n propuesta}
Esta entidad se encarga permite tanto la carga de registros, como dos diferentes tipos de lectura (completa y secuencial, o, de un �nico registro y direccionada). Para realizar el mapeo entre el objeto registro y los datos a almacenar se utilizan los serializadores en la secci�n \ref{sec:Serializadores}.

El serializador es configurado en cada archivo a manejar y realiza las dos conversiones necesarias: 
\begin{list}{+}{\setlength{\rightmargin}{\leftmargin}}
\item la tira de bytes leida en un objeto (Mapeo)
\item un objeto en una tira de bytes que ser� grabada (Serializaci�n)
\end{list}

El manejo de este archivo es simple, a medida que se agregan objetos serializa el objeto y lo agrega al �ltimo bloque. Este �ltimo bloque no se graba hasta que el tama�o del mismo no supere el tama�o de datos del bloque. Una vez que supera, se guardan todos los registros menos el �ltimo que se agreg� en el bloque que corresponda. Luego, este �ltimo registro agregado queda como el �nico registro que se encuentra en el �ltimo bloque.
