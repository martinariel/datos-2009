%%%%%%%%%%%%%%%%%%%%%%%%%%%%%%%%%%%%%%%%%%%%%%%%%%%%%%%%%%%%%%%%%%%%%%%%%%%%%%%
% Definici�n del tipo de documento.                                           %
% Posibles tipos de papel: a4paper, letterpaper, legalpapper                  %
% Posibles tama�os de letra: 10pt, 11pt, 12pt                                 %
% Posibles clases de documentos: article, report, book, slides                %
%%%%%%%%%%%%%%%%%%%%%%%%%%%%%%%%%%%%%%%%%%%%%%%%%%%%%%%%%%%%%%%%%%%%%%%%%%%%%%%
\documentclass[11pt, spanish, a4paper]{article}

%%%%%%%%%%%%%%%%%%%%%%%%%%%%%%%%%%%%%%%%%%%%%%%%%%%%%%%%%%%%%%%%%%%%%%%%%%%%%%%
% Los paquetes permiten ampliar las capacidades de LaTeX.                     %
%%%%%%%%%%%%%%%%%%%%%%%%%%%%%%%%%%%%%%%%%%%%%%%%%%%%%%%%%%%%%%%%%%%%%%%%%%%%%%%
\usepackage[spanish]{babel}     % Paquete para definir el idioma usado.
\usepackage[latin1]{inputenc}   % Define la codificaci�n de caracteres 
                                % (latin1 es ISO 8859-1)
%\usepackage[T1]{fontenc}        % Agrega caracteres extendidos al font
\usepackage{t1enc}
\usepackage{palatino}           % Cambia el font por omision a Palatino
\usepackage{graphicx}           % Paquete para inclusi�n de gr�ficos.
%%%%%%%%%%%%%%%%%%%%%%%%%%%%%%%%%%%%%%%%%%%%%%%%%%%%%%%%%%%%%%%%%%%%%%%%%%%%%%%

% T��tulo principal del documento.
\title{\textbf{The Speaker (Trabajo Pr�ctico Nro. 1)}}

% Informaci�n sobre los autores.
\author{    
            Juan Manuel Barrenche, \textit{Padr�n Nro. 86.152}               \\
            \texttt{ snipperme@gmail.com }                                   \\
            Mart��n Fern�ndez, \textit{Padr�n Nro. 88.171}                    \\
            \texttt{ tinchof@gmail.com }                                     \\
            Ra�l Lopez, \textit{Padr�n Nro. 88.430}                           \\
            \texttt{ rau\_carpo@hotmail.com }                                \\
            Marcos J. Medrano, \textit{Padr�n Nro. 86.729}                   \\
            \texttt{ marcosmedrano0@gmail.com }                              \\
            Federico Valido, \textit{Padr�n Nro. 82.490}                     \\
            \texttt{ fvalido@gmail.com }                                     \\ 
                                                                             \\
            \normalsize{Grupo Nro. 11 (YES) - 1er. Cuatrimestre de 2009}     \\
            \normalsize{75.06 Organizaci�n de Datos}                         \\
            \normalsize{Facultad de Ingenier�a, Universidad de Buenos Aires} \\
       }
\date{Domingo 5 de Abril de 2009}


% Comienzo del documento
\begin{document}

\maketitle                % Inserta el t�tulo.
\thispagestyle{empty}     % Quita el n�mero en la primer p�gina.

% Resumen que aparece en la primera p�gina (antes de la tabla de contenidos)
\begin{abstract}
Este parrafo deber�a contener una breve descripci�n del trabajo practico. No 
deber�a contener mas de 100 palabras.
En una introducci�n posterior se podr�a describir con m�s detalle los contenidos
de este trabajo.
Documento desarrollado en LATEX.
\end{abstract}

\newpage
\tableofcontents        % Comentar si no se desea la Table Of Content.
\newpage

% Inclusi�n de archivos externos
\section{Introducci�n}

En las siguientes secciones trataremos de explicar el funcionamiento integral de nuestra soluci�n para el sistema {\it ** The Speaker **}.

El problema planteado, para esta primera entrega, consta de dos partes o funcionalidades principales. El sistema debe ser capaz de cargar textos, durante esta carga, cada termino dentro del texto que no exista debe permitirle al usuario grabar el audio asociado al mismo. Es decir, que se registrar� esa palabra y se la asociar� al audio capturado por el micr�fono del usuario y que el mismo confirme. La otra funcionalidad importante es la de reproducci�n de un archivo completo. Esta opci�n recorrer� todos los terminos del archivo que el usuario elija y reproducir� su audio asociado. Los terminos desconocidos no ser�n tenidos en cuenta para la reproducci�n del audio.

La soluci�n fue implementada en Java. Intentando concentrarse en dejar la suficiente flexibilidad en el sistema para aplicar modificaciones para las siguientes entregas que requieren la ampliaci�n de funcionalidad en el programa y optimizaci�n de la recuperaci�n de la informaci�n manejada. Para lograr esto se realiz� una divisi�n del sistema en m�dulos, como se puede ver en la definici�n de la arquitectura, con responsabilidades claramente delimitadas.

La divisi�n de trabajos fue en base a estos m�dulos (descritos en la secci�n de arquitectura). Los diferentes m�dulos fueron implementados por diferentes miembros del grupo en colaboraci�n con los responsables de los m�dulos que tuvieran interacci�n con el modulo en cuesti�n. En una reuni�n inicial se bosquejaron las ideas para la soluci�n. Que luego fueron separadas en m�dulos y asignados a cada uno de los integrantes. 

Para el manejo de la informaci�n que administra  {\it ** The Speaker **} y la organizaci�n de los datos nos basamos en lo explicado durante las clases te�ricas dictadas en la materia.

La documentaci�n, exceptuando el manual del usuario, fue creado con \LaTeX.

\footnotetext[1]{Para la documentaci�n del usuario se recomienda leer el documento {\it user.pdf} que se encuentra en la ruta docs/user/}

         % Introducci�n
\section{Arquitectura del Sistema}

Se busco encontrar una soluci�n general, siempre que fuera posible, a los problemas particulares planteados. Para ello en primer lugar se dividi� el problema principal en varios subproblemas de alcance acotado.

\subsection{M�dulos}

Se definieron entonces los siguientes m�dulos:
\begin{itemize}
\item Parser: Se encarga de la normalizaci�n del texto y extracci�n de las diferentes palabras. Es utilizado por el programa principal, pidi�ndole el listado de palabras normalizadas de un archivo.
\item AudioService: Abstracci�n de las clases de manejo de audio proporcionada por la materia. Se incluye un InputStream para proporcionar a la biblioteca de audio que facilita la serializacion de ese InputStream. Es utilizado por el programa principal para la obtenci�n y reproducci�n de audios.
\item Buffers: Permiten el intercambio de datos con los archivos, funcionando como una memoria temporal donde se acumulan los datos leidos y los que se van a escribir. Son utilizados por el FileManager como herramienta para la lectura y escritura de datos en conjunto con los serializadores. Es la herramienta de intercambio de datos entre el FileManager y los Serializadores.
\item Serializers: permiten realizar la transformaci�n de los datos hacia un tipo unificado que pueda ser almacenado en los archivos y su posterior recuperaci�n. Brinda una interfaz unificada para que el FileManager pueda realizar la conversi�n de los datos recibidos. Intercambia los datos con el FileManager mediante el uso de Buffers. Se hicieron implementaciones gen�ricas (para los datos primitivos) e implementaciones customizadas para nuestros registros que usan auxiliarmente a las gen�ricas.
\item FileManager: Manejador de archivos de registros de longitud variable en bloques. Es una implementaci�n generica que se customiza indicando el tama�o de los bloques y el serializador a usar. Permite el encadenado de bloques de manera tal que un registro serializado que exceda un bloque pueda ser guardado. Utiliza auxiliarmente a los Buffers para la acumulaci�n de datos desde y hacia los archivos y para la interacci�n con los serializadores.
\item PersistenceService: Es, principalmente, una customizaci�n (dos en realidad) del FileManager para la interacci�n con los archivos de los requerimientos; esta customizaci�n implica la definici�n de los serializadores (dos tambi�n) adecuados para los registros a usar. Adem�s provee herramientas al principal para la realizaci�n a un nivel m�s alto de las acciones con los archivos requeridas.
\end{itemize}

\begin{figure}[!htp]
	\begin{center}
		\includegraphics[scale=0.60,natwidth=512pt, natheight=163pt]{img/Componentes.png}
	\end{center}
	\caption{Diagrama de \textbf{M�dulos}} 
\end{figure}

\subsection{Definici�n de los datos}

La definici�n de los datos sigue la propuesta por los requerimientos, con la salvedad de que, dado que se usaron archivos con manejo de bloques, el offset del archivo de palabras fue dividido en dos partes; la primera indica el n�mero de bloque que almacena el sonido correspondiente; la segunda, el n�mero de objeto dentro de ese bloque.
F�sicamente, los bloques contienen informaci�n de control extra para el encadenamiento de bloques ya mencionado, pero esto se explica mejor en la secci�n correspondiente al FileManager
          % Descripci�n general de la Arquitectura e introducci�n de cada Modulo
\section{Manejadores de Archivo}

Los datos que manejara **The Speaker** ten�an, como requerimiento t�cnico, que ser persistidos en dos archivos separados y deb�an utilizar la siguiente estructura l�gica.
\begin{enumerate}
	\item \texttt{((palabra)i, offset)}
	\item \texttt{(stream de audio)}
\end{enumerate}
El primero est� compuesto de la dupla palabra (que es un identificador) y offset/referencia. Simbolizando que para la {\it palabra} el audio se encuentra en {\it referencia}.
El segundo archivo contiene los streams de audio capturados.

\subsection{An�lisis}
Lo primero que se observa de ambos archivos es que sus registros son de longitud variable y que hay homogeneidad en los datos a almacenar, es decir, que en cada archivo se almacenan siempre los mismos tipos de datos. De manera que, ambos archivos, a pesar de tener naturalezas de datos diferentes requieren el mismo manejo. Por lo cual, admiten una misma soluci�n de manejo del archivo mientras que la misma se mantenga indepen de la naturaleza de los datos a almacenar. 

Por otra parte, las operaciones que se deben permitir son el agregado de registros y la consulta de los mismos. El agregado de registros no requiere, en ninguno de los dos casos, que se haga con un orden espec�fico. Mientras que la recuperaci�n de los datos, por otra parte, en el primer archivo debe poder ser secuencial (ya que se necesitan poder acceder a cada una de las palabras) y en el segundo caso debe poder accederse directamente a un registro que conozco su posici�n dentro del archivo. Por lo cual, estamos ante una organizaci�n secuencial de acceso relativo. Pudiendo recorrerse tanto secuencialmente como acceder a un registro espec�fico (si se conoce previamente su direcci�n).

\subsection{Soluci�n propuesta}
\paragraph{}
Se utilizar�n instancias de una clase llamada \textbf{VariableLengthFileManager} para abstraer a cada uno de los archivo. Esta clase define el comportamiento tanto, de la carga de registros, como de las dos formas diferentes de recuperaci�n de registros (completa y secuencial, y, de un �nico registro y direccionada). 

Para que la misma clase pueda persistir archivos con registros de diferente naturaleza se implementaron los serializadores  {\textit{(ver secci�n \ref{sec:Serializadores}.)}}
El serializador es configurado en cada archivo a manejar y realiza las dos conversiones necesarias: 
\begin{itemize}
	\item la tira de bytes leida en un objeto (Mapeo)
	\item un objeto en una tira de bytes que ser� grabada (Serializaci�n)
\end{itemize}

Esta clase no manejar� directamente el acceso al archivo f�sico si no que delegar� en un fino wrapper de la clase \textbf{RandomAccessFile} que se encargar� del manejo de los datos en bloque. {\it Ver figura \ref{fig:classVLFM}.}

\begin{figure}[!htp]
	\includegraphics[scale=0.35,natwidth=40pt, natheight=20pt]{img/ClassDiagramVariableLengthFileManager.png}
	\caption{Diagrama de clases del \textbf{VariableLengthFileManager}} 
	\label{fig:classVLFM}
\end{figure}

\paragraph{}
El manejo de este archivo es simple, a medida que se le solicita agregar objetos los serializa con el serializador con que fue configurado y los agrega al �ltimo bloque (esto no significa que se escriba en este momento). Si se le solicita alg�n objeto en particular, mediante la direcci�n del mismo, este manejador de archivo accede al bloque que indique la direcci�n y mapea los datos del objeto correcto utilizando el mismo serializador.

\subsubsection[Operaci�n de creaci�n]{Agregado de registro}
\paragraph{}
El agregado de registros, como se mencion� anteriormente, primero serializa el objeto y luego lo agrega al �ltimo bloque. Este �ltimo bloque siempre se encuentra cacheado ya que se graba (o regraba) cuando esa cach�, de �ltimo bloque, desborda, es decir, su tama�o supera el tama�o designado para datos del bloque. En ese momento se graban todos los registros que estaban en cach� menos el �ltimo agregado. Si, este �ltimo agregado, tuviera un tama�o mayor al tama�o de designado para datos del bloque el mismo es dividido en n bloques y todos esos bloques son grabados.
{\it Ver figura \ref{fig:sequenceVLFM_add}.}
\begin{figure}[!htp]
	\begin{center}
		\includegraphics[scale=0.4,natwidth=20pt,natheight=10pt]{img/SequenceDiagramVLFM_add.png}
	\end{center}
	\caption{Diagrama de secuencia que muestra el agregado de un registro} 
	\label{fig:sequenceVLFM_add}
\end{figure}

\subsubsection[Operaci�n de lectura secuencial]{Lectura de todos los datos}
\paragraph{}
Se implement� un iterador de todo el archivo que comienza en el bloque cero, mapea todos los datos de ese bloque a objetos y los va devolviendo de a uno. Luego de devolver todos los de ese bloque, pasa al siguiente bloque y realiza la misma operatori	a. Esto se repite hasta que no queden mas datos por hidratar. 
{\it Ver figura \ref{fig:sequenceVLFM_getAll}.}

\begin{figure}[!htp]
	\begin{center}
		\includegraphics[scale=0.4,natwidth=20pt,natheight=10pt]{img/SequenceDiagramVLFM_getAll.png}
	\end{center}
	\caption{Diagrama de secuencia que muestra la iteraci�n sobre todos los registros} 
	\label{fig:sequenceVLFM_getAll}
\end{figure}

\subsubsection[Operaci�n de lectura aleatoria]{Lectura de un objeto dada una direcci�n}
El manejador de archivo lee el bloque desde el archivo f�sico (excepto que el mismo est� en cach�), y luego pasa los datos leidos por el serializador. Luego busca entre los objetos creados el que tenga la posici�n indicada por la direcci�n para poder devolverlo a qui�n se lo haya solicitado.
{\it Ver figura \ref{fig:sequenceVLFM_get}.}

\begin{figure}[!htp]
	\begin{center}
		\includegraphics[scale=0.4,natwidth=20pt, natheight=10pt]{img/SequenceDiagramVLFM_get.png}
	\end{center}
	\caption{Diagrama de secuencia que muestra la recuperaci�n de un �nico registro} 
	\label{fig:sequenceVLFM_get}
\end{figure}

\subsection{Detalles t�cnicos}
Los bloques cuentan con la siguiente informaci�n de control. Los �ltimos 2 bytes indican (almacenado como un Short signado) la cantidad de registros enteros que posee el bloque (esto se utiliza al momento de hidratar, para no intentar hidratar mas registros de los que se encontraban almacenados), para el caso que el registro est� en m�ltiples bloques este valor se marca en cero y se toman los 8 bytes anteriores para indicar la posici�n del pr�ximo bloque que contiene datos del mismo registro.

Se implementaron 2 cach�s, muy b�sicas, la primera, ya fue mencionada, contiene el bloque actual donde se est�n agregando registros. La segunda contiene el �ltimo bloque leido del disco (para disminuir accesos a disco)


          % Manejadores de Archivos
\section{Buffers de Entrada y Salida}

texto

\subsection{Subsecci�n}

texto

\subsubsection{SubSubSecci�n}

texto
        % Buffers de Entrada y Salida
\section{Serializadores}

texto

\subsection{Subsecci�n}

texto

\subsubsection{SubSubSecci�n}

texto
    % Serializadores
\section{Servicios de Persistencia}

La persistencia ten�a como requerimiento t�cnico que los datos que maneja **The Speaker** fueran persistidos en dos archivos separados. En esta primera entrega el requerimiento era que fueran estrictamente dos archivos con la siguiente estructura l�gica definida.
\newcounter{archivo}
\begin{list}{\addtocounter{archivo}{1}\arabic{archivo}}
{
\setlength{\rightmargin}{\leftmargin}
}
\item \texttt{((palabra)i, offset)}
\item \texttt{(stream de audio)}
\end{list}
El primero est� compuesto de la dupla palabra (que es un identificador) y offset/referencia. Simbolizando que para la {\it palabra} el audio se encuentra en {\it referencia}.
El segundo archivo contiene los streams de audio capturados.

\subsection{An�lisis}
Lo primero que se observa de ambos archivos es que sus registros son de longitud variable y que hay homogeneidad en los datos a almacenar, es decir, que en cada archivo se almacenan siempre los mismos tipos de datos. De manera que, ambos archivos, a pesar de tener naturalezas de datos diferentes requieren el mismo manejo. Por lo cual, admiten una misma soluci�n de manejo del archivo mientras que la misma se mantenga indepen de la naturaleza de los datos a almacenar. 

Por otra parte, las operaciones que se deben permitir son el agregado de registros y la consulta de los mismos. El agregado de registros no requiere, en ninguno de los dos casos, que se haga con orden alguno. Mientras que la recuperaci�n de los datos, por otra parte, en el primer archivo debe poder ser secuencial (ya que se necesitan poder acceder a cada una de las palabras) y en el segundo caso debe poder accederse directamente a un registro que conozco su posici�n dentro del archivo. Por lo cual, estamos ante una organizaci�n secuencial de acceso relativo. Pudiendo recorrerse tanto secuencialmente como acceder a un registro espec�fico (si se conoce previamente su direcci�n). La implementaci�n de esta funcionalidad se hizo en la clase  {\textbf{VariableLengthFileManager}.}

\subsection{Soluci�n propuesta}
Esta entidad se encarga permite tanto la carga de registros, como dos diferentes tipos de lectura (completa y secuencial, o, de un �nico registro y direccionada). Para realizar el mapeo entre el objeto registro y los datos a almacenar se utilizan los serializadores en la secci�n \ref{sec:Serializadores}.

El serializador es configurado en cada archivo a manejar y realiza las dos conversiones necesarias: 
\begin{list}{+}{\setlength{\rightmargin}{\leftmargin}}
\item la tira de bytes leida en un objeto (Mapeo)
\item un objeto en una tira de bytes que ser� grabada (Serializaci�n)
\end{list}

El manejo de este archivo es simple, a medida que se agregan objetos serializa el objeto y lo agrega al �ltimo bloque. Este �ltimo bloque no se graba hasta que el tama�o del mismo no supere el tama�o de datos del bloque. Una vez que supera, se guardan todos los registros menos el �ltimo que se agreg� en el bloque que corresponda. Luego, este �ltimo registro agregado queda como el �nico registro que se encuentra en el �ltimo bloque.
   % Servicios de Persistencia

\end{document}
