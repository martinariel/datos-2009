%%%%%%%%%%%%%%%%%%%%%%%%%%%%%%%%%%%%%%%%%%%%%%%%%%%%%%%%%%%%%%%%%%%%%%%%%%%%%%%
% Definici�n del tipo de documento.                                           %
% Posibles tipos de papel: a4paper, letterpaper, legalpapper                  %
% Posibles tama�os de letra: 10pt, 11pt, 12pt                                 %
% Posibles clases de documentos: article, report, book, slides                %
%%%%%%%%%%%%%%%%%%%%%%%%%%%%%%%%%%%%%%%%%%%%%%%%%%%%%%%%%%%%%%%%%%%%%%%%%%%%%%%
\documentclass[a4paper,10pt]{article}

%%%%%%%%%%%%%%%%%%%%%%%%%%%%%%%%%%%%%%%%%%%%%%%%%%%%%%%%%%%%%%%%%%%%%%%%%%%%%%%
% Los paquetes permiten ampliar las capacidades de LaTeX.                     %
%%%%%%%%%%%%%%%%%%%%%%%%%%%%%%%%%%%%%%%%%%%%%%%%%%%%%%%%%%%%%%%%%%%%%%%%%%%%%%%

% Paquete para inclusi�n de gr�ficos.
\usepackage{graphicx}

% Paquete para definir la codificaci�n del conjunto de caracteres usado
% (latin1 es ISO 8859-1).
\usepackage[latin1]{inputenc}

% Paquete para definir el idioma usado.
\usepackage[spanish]{babel}


% T�tulo principal del documento.
\title{\textbf{The Speaker (Trabajo Pr�ctico Nro. 1)}}

% Informaci�n sobre los autores.
\author{    
            Juan Manuel Barrenche, \textit{Padr�n Nro. 86.152}              \\
            \texttt{ snipperme@gmail.com }                                  \\
            Mart�n Fern�ndez, \textit{Padr�n Nro. 88.171}                   \\
            \texttt{ tinchof@gmail.com }                                    \\
            R�l Lopez, \textit{Padr�n Nro. 88.430}                          \\
            \texttt{ rau\_carpo@hotmail.com }                               \\
            Marcos J. Medrano, \textit{Padr�n Nro. 86.729}                  \\
            \texttt{ marcosmedrano0@gmail.com }                             \\
            Federico Valido, \textit{Padr�n Nro. 82.490}                    \\
            \texttt{ fvalido@gmail.com }                                    \\ 
                                                                            \\
            \normalsize{Grupo Nro. 11 (YES) - 1er. Cuatrimestre de 2009}    \\
            \normalsize{75.06 Organizaci�n de Datos}                        \\
            \normalsize{Facultad de Ingenier�a, Universidad de Buenos Aires}\\
       }
\date{DIA de MES de 2009}

\begin{document}

% Inserta el t�tulo.
\maketitle

% Quita el n�mero en la primer p�gina.
\thispagestyle{empty}

% Resumen
\begin{abstract}
Este parrafo deber�a contener una breve descripci�n del trabajo practico. No 
deber�a contener mas de 100 palabras.
En una introducci�n posterior se podr� describir con m�s detalle los contenidos
de este trabajo.
Documento desarrollado en \LaTeX
\end{abstract}

\newpage
\tableofcontents 
\newpage

\section{Introducci�n}

Escribir aqu� la Introducci�n

\section{Secci�n de Ejemplo}

\subsection{Subsecci�n de Ejemplo}

La siguiente es una lista de ejemplo: 
\begin{itemize}
\item Elemento 1
\item Elemento 2
\item Elemento 3
\end{itemize}

\subsection{Otra Subsecci�n}

Ejemplo de \texttt{tipograf�a monospace}

Ejemplo de texto en verbatim:
\begin{verbatim}
$ latex miarchivo.tex
\end{verbatim}

\section{Otra Secci�n}

Aqu� vemos una nota al pie de p�gina \footnote{Esto aparece abajo}
\subsection{Known Bugs (Bugs conocidos)}

% Descomentar el siguiente codigo para poner una im�gen centrada

%\begin{figure}[!htp]
%\begin{center}
%\includegraphics[scale=0.45, natwidth=50pt, natheight=20pt] {imagen.png}
%\end{center}
%\caption{Probando la entrada estandar.} \label{fig003}
%\end{figure}

\section{Conclusiones}

Escribir conclusiones aqu�

% Citas bibliogr�ficas.
\begin{thebibliography}{99}

% Escribir las citas bibliogr�ficas aqu�:
% Descomentar las siguientes l�nea para tener un ejemplo:

%\bibitem{WIK00} Wikipedia,  ``Articulo`` , http://es.wikipedia.org/wiki/articulo.
%\bibitem{REF00} Autor,  ``Libro`` , referencia.

\end{thebibliography}

\end{document}
